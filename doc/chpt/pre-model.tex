
\subsection{From spatial displacement to Q-ball imaging}

The basic concept of diffusion MRI is as follows:
A gradient is added twice to the magnetic field, each time for a very short
time interval $t_e$ (encoding time).
A larger time interval $t_d \gg t_e$ is left between the two pulses.
The measurement takes place after the second pulsed gradient and the measured
signal strength corresponds to the diffusion that has taken place during the
time interval $t_d$ in the direction of the applied gradient.
The diffusion taking place during $t_e$ is neglected.

According to (Cory, 1990), the measured signal \emph{at a fixed voxel}
depending on the applied gradient $g$ is given by 
\begin{align}
    E(q) &:= \int_{\IR^3} P(r) e^{i \, q \cdot r} \,dr,
\end{align}
where
\begin{align}
    q &:= \gamma t_e g, \\
    P(r) &:= \int_{\IR^3} \rho(x_0) P(x_0 + r|x_0) \,dx_0, \\
    \rho(x_0) &: \text{initial concentration at $x_0$}, \\
    P(x|x_0) &: \text{%
        probability that a spin initially ar $x_0$
        will move to $x$ (during $t_d$)%
    }, \\
    \gamma &: \text{magnetogyric ratio}, \\
    g &: \text{magnetic field gradient strength}.
\end{align}
The Fourier relationship is attributed to (Stejskal, Tanner, 1965).

For a complete reconstruction of the displacement profile $P(r)$ at the fixed
voxel, a lot of samples are needed to cover the whole three-dimensional Cartesian
lattice, which takes a really long time (more than 60 minutes) and poses
strong (hardware) demands on the magnetic field gradients.
This method is known as q-space imaging (QSI), diffusion spectrum imaging (DSI),
diffusion displacement imaging or dynamic NMR microscopy.

In DTI, as an a priori model assumption, the diffusion is assumed to take place
in a single direction only, restricting displacement profiles to Gaussian
diffusion.
To fit data to this model, only 6 measurements are necessary.
However, through experiments with QSI, it is known that the assumption of
monodirectional diffusion is not accurate for at least 30 percent of the voxels,
because the voxel size is pretty large.
DTI performs very badly in these voxels, generally pointing in the mean diffusion
direction.

In HARDI, this is tackled by restricting to gradient directions on a
fixed sphere in $q$-space so that we typically sample $E$ for 150--200 points
on a sphere $|q| \equiv const$.

Given the HARDI measurements, different reconstruction schemes have been
applied.
Q-ball imaging (Tuch, 2004) was the first technique for a model-free inversion.
It aims at reconstructing not $P(r)$ but only the radial marginals, or
orientation distribution functions (ODF),
\[
    u(x) := \int_0^\infty P(rx) r^2 \,dr,
\]
for unit direction vectors $x$.

\subsection{Q-ball imaging and Funk-Radon transform}

As a consequence of the microscopic detailed balance, $P(r)$ is assumed to have
a reflection symmetry $P(r) = P(-r)$.
Then, the relationship, for $|x| = 1$ and symmetric function $f$, 
\[
    \int_0^\infty f(rx) \,dr = \frac{1}{8\pi^2} \int_{x^\perp} \hat f (q) dq
\]
is used to derive the formula
\[
    u(x) = -\frac{1}{8\pi^2} \int_{x^\perp} \Delta E(q) dq,
\]
and this again can be written in spherical coordinates as
\[
    u(x) = \frac{1}{4\pi} - \frac{1}{8\pi^2} \int_0^{2\pi}\int_0^\infty
        \frac{1}{r} \Delta_b E(rx)
    \,drd\phi,
\]
where $\Delta_b$ is the Laplace-Beltrami operator.
The monoexponential model due to (Aganj, 2010),
\[
    E(rx) \approx E(r_0 x)^{\frac{r^2}{r_0^2}},
\]
allows to write this using the Funk-Radon transform,
\[
    u(x) = \frac{1}{4\pi} - \frac{1}{16\pi^2} \FRT\left[
        \Delta_b \log(-\log\tilde E)
    \right](x),
\]
where $\tilde E(x) := E(r_0 x)$ and
\[
    \FRT[f](x) := \int_{\IS^2 \cap x^\perp} f(z) \,dz.
\]
A similar formula can be derived for the case where the samples are taken on
more than one $q$-shell.

\subsection{Where spherical harmonics come into play}

The explicit formula allows for a voxel by voxel reconstruction by representing
the signal in the basis of spherical harmonic functions,
\begin{align}\label{eq:shm-fitting}
    \log(-\log\tilde E(x)) = \sum_j c_j Y_j(x).
\end{align}
This basis is particularly well-suited to this problem, because spherical
harmonics are Eigenvectors of the Laplace-Beltrami operator as well as of the
Funk-Radon-transform.
(Aganj, 2010) directly applies the spherical harmonics fitting
\eqref{eq:shm-fitting} voxel by voxel using a minimum square error scheme.
After that the reconstruction of $u$ is a simple multiplication by the
corresponding eigenvalues.

