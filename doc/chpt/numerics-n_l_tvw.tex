
\subsection{Implementing a quadratic data term}

\paragraph{Saddle point form}

Using a quadratic data term, the problem's saddle point form reads
\begin{align*}
    \min_{u} \max_{p,g} \quad
        & \frac{1}{2} \langle u-f, u-f \rangle_b + \langle Du, p \rangle_b \\
    \text{s.t.}\quad 
        & u^i \geq 0, ~\langle u^i, b \rangle = 1 ~\forall i, \\
        & A^j g^{ij}_t = B^j P^j p^i_t ~\forall i,j,t, \\
        & \|g^{ij}\|_{\sigma,\infty} \leq \lambda ~\forall i,j
\end{align*}
or, using more variables and less constraints,
\begin{align*}
    \min_{u,w} \max_{p,g,q} \quad
        & \frac{1}{2} \langle u-f, u-f \rangle_b
            + \langle Du, p \rangle_b \\
        &\quad + \sum_{i,j,t} \langle w^{ij}_t, A^j g^{ij}_t - B^j P^j p^i_t \rangle
            + \sum_{i} q^i \cdot (b^T u^i - 1) \\
    \text{s.t.}\quad 
        & u^i \geq 0, ~\|g^{ij}\|_{\sigma,\infty} \leq \lambda ~\forall i,j.
\end{align*}
Here, we denote the Schatten-$p$-norms by $\|\cdot\|_{\sigma,p}$.

\paragraph{Primal and dual objectives}

Accordingly, the primal formulation of the problem is
\begin{align*}
    \min_{u,w} \quad
        & \frac{1}{2} \langle u-f, u-f \rangle_b 
            + \lambda \sum_{i,j} \| A^{jT} w^{ij} \|_{\sigma,1} \\
    \text{s.t.}\quad 
        & u^i \geq 0, ~\langle u^i, b \rangle = 1 ~\forall i, \\
        & b_k (\partial_t u)_k^i = \sum_j (P^{jT}B^{jT}w^{ij}_t)_{k} ~\forall i,k,t
\end{align*}
and the dual formulation is
\begin{align*}
    \max_{p,g,q} \quad
        & -\sum_i q^i
            + \sum_{i,k} \frac{b_k}{2} \left [
                \left(f_k^i\right)^2
                - \min\left(0, q^i - (\divergence{p})_k^i - f_k^i\right)^2
            \right ] \\
    \text{s.t.}\quad 
        & \|g^{ij}\|_{\sigma,\infty} \leq \lambda ~\forall i,j, \\
        & A^j g^{ij}_t = B^j P^j p^i_t ~\forall i,j,t.
\end{align*}

\paragraph{Proximal mappings}

We rewrite the saddle point form as follows:
\begin{align*}
    \min_{u,w} \max_{p,g,q} \quad
        & G(u,w) + \langle K(u,w), (p,g,q) \rangle - F^*(p,g,q),
\end{align*}
where, writing $\beta = \diag(b)$,
\begin{align*}
    G(u,w) &= \frac{1}{2} \langle u-f, u-f \rangle_b
        + \delta_{\{u \geq 0\}}, \\
    F^*(p,g,q) &= \sum_{i} q^i 
        + \sum_{i,j} \delta_{\{\|g^{ij}\|_{\sigma,\infty} \leq \lambda\}}, \\
    K(u,w) &= (\beta Du - \sum_j P^{jT}B^{jT}w^{j}, A^T w, b^T u), \\
    K^*(p,g,q) &= (q \otimes b - \beta \divergence{p}, A g - PBp).
\end{align*}

For a first-order approach we use the proximal mappings
\begin{align*}
    \Prox_{\sigma F*}(\bar{p},\bar{g},\bar{q})
    &= \argmin_{p,g,q} \left\{
        \frac{\|(p,g,q)-(\bar{p},\bar{g},\bar{q})\|^2}{2\sigma} + F^*(p,g,q)
    \right\} \\
    &= (\bar{p},\proj_{\lambda,\infty}(\bar{g}),\bar{q}-\sigma e),
\end{align*}
where $e = (1,\dots,1)$, and
\begin{align*}
    \Prox_{\tau G}(\bar{u},\bar{w})
    &= \argmin_{u,w} \left\{
        \frac{\|(u,w)-(\bar{u},\bar{w})\|^2}{2\tau} + G(u,w)
    \right\} \\
    &= \left(
        \max\left(0,(I+\tau \beta)^{-1}(\bar{u} + \tau \beta f)\right),
        \bar{w}
    \right).
\end{align*}

\paragraph{The algorithm}

Now we have everything in place to formulate the algorithm:
\begin{align*}
    (p^{k+1},g^{k+1},q^{k+1}) &= \Prox_{\sigma F^*}(
        (p^{k},g^{k},q^{k}) + \sigma K(\bar{u}^{k},\bar{w}^{k})
    ), \\
    (u^{k+1},w^{k+1}) &= \Prox_{\tau G}(
        (u^{k},w^{k})-\tau K^*(p^{k+1},g^{k+1},q^{k+1})
    ), \\
    \bar{u}^{k+1} &= u^{k+1} + \theta(u^{k+1}-u^{k}), \\
    \bar{w}^{k+1} &= w^{k+1} + \theta(w^{k+1}-w^{k}).
\end{align*}

