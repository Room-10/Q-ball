
\subsection{Implementing a Wasserstein data term}

\paragraph{Saddle point form}

Using a Wasserstein data term, the problem's saddle point form reads
\begin{align*}
    \min_{u} \max_{p,g} \quad
        & W^1(u,f) + \langle Du, p \rangle_b \\
    \text{s.t.}\quad 
        & u^i \geq 0, ~\langle u^i, b \rangle = 1 ~\forall i, \\
        & A^j g^{ij}_t = B^j P^j p^i_t ~\forall i,j,t, \\
        & \|g^{ij}\|_{\sigma,\infty} \leq \lambda ~\forall i,j
\end{align*}
or, using more variables and less constraints,
\begin{align*}
    \min_{\mat{x}} \max_{\mat{y}} \quad
        & \langle u-f, p_0 \rangle_b
            + \langle Du, p \rangle_b
            + \sum_{i,j} \langle w_0^{ij}, A^j g_0^{ij} - B^j P^j p_0^i \rangle
            \\
        &\quad + \sum_{i,j,t} \langle w^{ij}_t, A^j g^{ij}_t - B^j P^j p^i_t \rangle
            + \sum_{i} q^i \cdot (b^T u^i - 1) \\
    \text{s.t.}\quad 
        & u^i \geq 0,
            ~\|g^{ij}\|_{\sigma,\infty} \leq \lambda,
            ~\|g_0^{ij}\|_{2} \leq 1 ~\forall i,j,
\end{align*}
writing $\mat{x} = (u,w,w_0)$, $\mat{y} = (p,g,q,p_0,g_0)$ (TODO: Use this 
abbreviated notation everywhere!).
We denote the Schatten-$p$-norms by $\|\cdot\|_{\sigma,p}$.

\paragraph{Primal and dual objectives}

Accordingly, the primal formulation of the problem is
\begin{align*}
    \min_{u,w,w_0} \quad
        & \sum_{i,j} \| A^{jT} w_0^{ij} \|_{2}
            + \lambda \sum_{i,j} \| A^{jT} w^{ij} \|_{\sigma,1} \\
    \text{s.t.}\quad 
        & u^i \geq 0, ~\langle u^i, b \rangle = 1 ~\forall i, \\
        & b_k (\partial_t u)_k^i = \sum_j (P^{jT}B^{jT}w^{ij}_t)_{k} ~\forall i,k,t, \\
        & b_k (u - f)_k^i = \sum_j (P^{jT}B^{jT}w_0^{ij})_{k} ~\forall i,k.
\end{align*}
and the dual formulation is
\begin{align*}
    \max_{p,g,q,p_0,g_0} \quad
        & - \langle f, p_0 \rangle_b - \sum_i q^i \\
    \text{s.t.}\quad 
        & \|g^{ij}\|_{\sigma,\infty} \leq \lambda,~ 
          \|g_0^{ij}\|_{2} \leq 1 ~\forall i,j, \\
        & A^j g^{ij}_t = B^j P^j p^i_t,~
          A^j g_0^{ij} = B^j P^j p_0^i ~\forall i,j,t, \\
        & b_k p_0^{ki} - b_k (\divergence p^k)^i + q^i b^k \geq 0 ~\forall i, k.
\end{align*}

\paragraph{Proximal mappings}

We rewrite the saddle point form as follows:
\begin{align*}
    \min_{u,w,w_0} \max_{p,g,\dots} \quad
        & G(u,w,w_0)
          + \langle K(u,w,w_0), (p,g,\dots) \rangle
          - F^*(p,g,\dots),
\end{align*}
where, writing $\beta = \diag(b)$,
\begin{align*}
    G(u,w,w_0) &= \delta_{\{u \geq 0\}}, \\
    F^*(p,g,\dots) &= \langle f, p_0 \rangle_b + \sum_{i} q^i 
        + \sum_{i,j} \delta_{\{
            \|g^{ij}\|_{\sigma,\infty} \leq \lambda,
            \|g_0^{ij}\|_{2} \leq 1
        \}}, \\
    K(u,w,w_0) &= (
        \beta Du - \sum_j P^{jT}B^{jT}w^{j},
        A^T w,
        b^T u,
        \beta u - \sum_j P^{jT}B^{jT}w_0^{j},
        A^T w_0
    ), \\
    K^*(p,g,\dots) &= (
        \beta p_0 + q \otimes b - \beta \divergence{p},
        A g - PBp,
        A g_0 - PB p_0,
    ).
\end{align*}

For a first-order approach we use the proximal mappings
\begin{align*}
    \Prox_{\sigma F*}(\bar{p},\bar{g},\dots)
    &= \argmin_{p,g,\dots} \left\{
        \frac{
            \|(p,g,\dots)-(\bar{p},\bar{g},\dots)\|^2
        }{2\sigma} + F^*(p,g,\dots)
    \right\} \\
    &= (
        \bar{p},
        \proj_{\lambda,\infty}(\bar{g}),
        \bar{q}-\sigma e,
        \bar{p_0} - \sigma \beta f,
        \proj_{1,2}(\bar{g_0})
    ),
\end{align*}
where $e = (1,\dots,1)$, and
\begin{align*}
    \Prox_{\tau G}(\bar{u},\bar{w},\bar{w_0})
    &= \argmin_{u,w,w_0} \left\{
        \frac{
            \|(u,w,w_0)-(\bar{u},\bar{w},\bar{w_0})\|^2
        }{2\tau} + G(u,w,w_0)
    \right\} \\
    &= \left(
        \max(0,\bar{u}),
        \bar{w},
        \bar{w_0}
    \right).
\end{align*}

\paragraph{The algorithm}

Now we have everything in place to formulate the algorithm:
\begin{align*}
    (p^{k+1},g^{k+1},\dots) &= \Prox_{\sigma F^*}(
        (p^{k},g^{k},\dots)
        + \sigma K(\bar{u}^{k},\bar{w}^{k},\bar{w_0}^{k})
    ), \\
    (u^{k+1},w^{k+1},w_0^{k+1}) &= \Prox_{\tau G}(
        (u^{k},w^{k},w_0^{k})-\tau K^*(p^{k+1},g^{k+1},\dots)
    ), \\
    \bar{u}^{k+1} &= u^{k+1} + \theta(u^{k+1}-u^{k}), \\
    \bar{w}^{k+1} &= w^{k+1} + \theta(w^{k+1}-w^{k}), \\
    \bar{w_0}^{k+1} &= w_0^{k+1} + \theta(w_0^{k+1}-w_0^{k}).
\end{align*}

