
\subsection{Implementing TV-W1 for HARDI data}

\paragraph{Saddle point form}

Using the HARDI data instead of a Q-ball reconstruction, we can reconstruct
and regularize at the same time.
In addition to the spherical harmonics sampling Matrix $\Psi \in \IR^{l,l'}$,
we need the reconstruction matrix $M \in \IR^{l',l'}$.
If we set $f = \log(-\log(E))$, where $E$ are the HARDI data, $M$ is a diagonal
matrix.
For now, we combine a quadratic data term with our TV-W1-regularizer.
The problem's saddle point form then reads
\begin{align*}
    \min_{u_1,u_2,v} \max_{p,g} \quad
        & \frac{1}{2} \langle u_2 - f, u_2 - f \rangle_b + \langle Du_1, p \rangle_b \\
    \text{s.t.}\quad 
        & \Psi M v^i = u_2^i, ~\Psi v^i = u_1^i ~\forall i, \\
        & u_1^i \geq 0, ~\langle u_1^i, b \rangle = 1 ~\forall i, \\
        & A^j g^{ij}_t = B^j P^j p^i_t ~\forall i,j,t, \\
        & \|g^{ij}\|_{\sigma,\infty} \leq \lambda ~\forall i,j
\end{align*}
or, using more variables and less constraints,
\begin{align*}
    \min_{u_1,u_2,v,w} \max_{p,g,q_0,q_1,q_2} \quad
        & \frac{1}{2} \langle u_2 - f, u_2 - f \rangle_b
            + \langle Du_1, p \rangle_b
            + \sum_{i,j,t} \langle w^{ij}_t, A^j g^{ij}_t - B^j P^j p^i_t \rangle \\
        &\quad + \sum_{i} q_0^i \cdot (b^T u_1^i - 1)
            + \sum_{i} \langle q_1^i, \Psi v^i - u_1^i \rangle
            + \sum_{i} \langle q_2^i, \Psi M v^i - u_2^i \rangle \\
    \text{s.t.}\quad 
        & u_1^i \geq 0, ~\|g^{ij}\|_{\sigma,\infty} \leq \lambda ~\forall i,j.
\end{align*}
Here, we denote the Schatten-$p$-norms by $\|\cdot\|_{\sigma,p}$.

\paragraph{Primal and dual objectives}

Accordingly, the primal formulation of the problem is
\begin{align*}
    \min_{u_1,u_2,v,w} \quad
        & \frac{1}{2} \langle u_2 - f, u_2 - f \rangle_b 
            + \lambda \sum_{i,j} \| A^{jT} w^{ij} \|_{\sigma,1} \\
    \text{s.t.}\quad 
        & \Psi M v^i = u_2^i, ~\Psi v^i = u_1^i ~\forall i, \\
        & u_1^i \geq 0, ~\langle u_1^i, b \rangle = 1 ~\forall i, \\
        & b_k (\partial_t u_1)_k^i = \sum_j (P^{jT}B^{jT}w^{ij}_t)_{k} ~\forall i,k,t
\end{align*}
and the dual formulation is
\begin{align*}
    \max_{p,g,q_0,q_1,q_2} \quad
        & -\sum_i q_0^i+ \sum_{i,k} \frac{b_k}{2} \left [
                \left(f_k^i\right)^2
                - \left(b_k^{-1} q_2^{ik} + f_k^i\right)^2
            \right ] \\
    \text{s.t.}\quad 
        & \|g^{ij}\|_{\sigma,\infty} \leq \lambda ~\forall i,j, \\
        & A^j g^{ij}_t = B^j P^j p^i_t ~\forall i,j,t, \\
        & q_0^i b^k - b_k (\divergence p^k)^i - q_1^{ki} \geq 0 ~\forall i, k, \\
        & \Psi^T q_1^i = -M \Psi^T q_2^i ~\forall i.
\end{align*}

\paragraph{Proximal mappings}

We rewrite the saddle point form as follows:
\begin{align*}
    \min_{u_1,u_2,v,w} \max_{p,g,\dots} \quad
        & G(u_1,u_2,v,w) + \langle K(u_1,u_2,v,w), (p,g,\dots) \rangle - F^*(p,g,\dots),
\end{align*}
where, writing $\beta = \diag(b)$,
\begin{align*}
    G(u_1,u_2,v,w) &= \frac{1}{2} \langle u_2-f, u_2-f \rangle_b
        + \delta_{\{u_1 \geq 0\}}, \\
    F^*(p,g,\dots) &= \sum_{i} q_0^i 
        + \sum_{i,j} \delta_{\{\|g^{ij}\|_{\sigma,\infty} \leq \lambda\}}, \\
    K(u_1,u_2,v,w) &= (
        \beta Du_1 - \sum_j P^{jT}B^{jT}w^{j},
        A^T w,
        b^T u_1,
        \Psi v^i - u_1^i,
        \Psi M v^i - u_2^i
    ), \\
    K^*(p,g,\dots) &= (
        q \otimes b - \beta \divergence{p} - q_1,
        -q_2,
        \Psi^T q_1^i + M \Psi^T q_2^i,
        A g - PBp
    ).
\end{align*}

For a first-order approach we use the proximal mappings
\begin{align*}
    \Prox_{\sigma F*}(\bar{p},\bar{g},\dots)
    &= \argmin_{p,g,\dots} \left\{
        \frac{\|
            (p,g,\dots)-(\bar{p},\bar{g},\dots)
        \|^2}{2\sigma} + F^*(p,g,\dots)
    \right\} \\
    &= (\bar{p},\proj_{\lambda,\infty}(\bar{g}),\bar{q_0}-\sigma e,\bar{q_1},\bar{q_2}),
\end{align*}
where $e = (1,\dots,1)$, and
\begin{align*}
    \Prox_{\tau G}(\bar{u_1},\bar{u_2},\bar{v},\bar{w})
    &= \argmin_{u_1,u_2,v,w} \left\{
        \frac{\|
            (u_1,u_2,v,w)-(\bar{u_1},\bar{u_2},\bar{v},\bar{w})
        \|^2}{2\tau} + G(u_1,u_2,v,w)
    \right\} \\
    &= \left(
        \max(0,\bar{u_1}),
        (I+\tau \beta)^{-1}(\bar{u_2} + \tau \beta f),
        \bar{v},
        \bar{w}
    \right).
\end{align*}

\paragraph{The algorithm}

Now we have everything in place to formulate the algorithm:
\begin{align*}
    (p^{k+1},g^{k+1},\dots) &= \Prox_{\sigma F^*}(
        (p^{k},g^{k},\dots)
        + \sigma K(\bar{u_1}^{k},\bar{u_2}^{k},\bar{v}^{k},\bar{w}^{k})
    ), \\
    (u_1^{k+1},u_2^{k+1},v^{k+1},w^{k+1}) &= \Prox_{\tau G}(
        (u_1^{k},u_2^{k},v^{k},w^{k})
        - \tau K^*(p^{k+1},g^{k+1},\dots)
    ), \\
    \bar{u_1}^{k+1} &= u_1^{k+1} + \theta(u_1^{k+1}-u_1^{k}), \\
    \bar{u_2}^{k+1} &= u_2^{k+1} + \theta(u_2^{k+1}-u_2^{k}), \\
    \bar{v}^{k+1} &= v^{k+1} + \theta(v^{k+1}-v^{k}), \\
    \bar{w}^{k+1} &= w^{k+1} + \theta(w^{k+1}-w^{k}).
\end{align*}

