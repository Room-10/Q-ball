
\subsection{Preprocessed W1-TV restoration (SSVM 2017)}

In our SSVM paper, we used the following variational approach,
\[
    \inf_{u:\Omega \to \IP(\IS^2)}
        \int_\Omega W_1(f_x,u_x) \dd x + \lambda \TV_{W_1}(u),
\]
to restore the ODF $u$ from the result $f$ of the voxel by voxel reconstruction
based on (Aganj, 2010).\\

\vspace{1cm}
\hspace{1cm}\vdots~~ Describe $\TV_{W_1}$ and $W_1$ etc. ~~\vdots
\vspace{1cm}

Alternatively, you could do a spherical regularization by assuming that your
minimizer has a spherical harmonics representation $v = (c_j)_{j \in \IN}$
such that $u = \Psi v = \sum_j c_j Y_j$ (i.~e. $\Psi$ is the spherical harmonics
sampling operator):
\[
    \inf_{v \in \IR^\IN}
        \int_\Omega W_1(f_x, (\Psi v)_x) \dd x + \lambda \TV_{W_1}((\Psi v)),
\]

\subsection{Direct W1-TV reconstruction}

Actually, the input $f$ in the above approach was not directly the output of
(Aganj, 2010) since $f$ was only sampled on a triangulation of the sphere
whereas the reconstruction scheme returns a spherical harmonics representation.
However, it was not directly possible to use the spherical harmonics
representation because the Kantorovich-Rubinstein formulation of the
Wasserstein metric involves a derivative constraint on the sphere that cannot
be written in spherical harmonics equally well.
Our new approach combines HARDI fitting with the new spatial regularizer
\[
    \inf_{v \in \IR^\IN}
        \| \Psi A^{-1}v - F \|_{L^2}^2 + \lambda \TV_{W_1}(\Psi v),
\]
where $F(x) := \log(-\log\tilde E(x))$ and $A$ is the diagonal operator involving
the eigenvalues with respect to $\FRT$ and $\Delta_b$.

In a next stept we will replace the data term by bounds derived from
confidence intervals:
\[
    \inf_{v \in \IR^\IN}
        \| (\Psi A^{-1}v - F^u)_+ \|_{L^1}  + \| (F^l - \Psi A^{-1}v)_+ \|_{L^1}
        + \lambda \TV_{W_1}(\Psi v).
\]
