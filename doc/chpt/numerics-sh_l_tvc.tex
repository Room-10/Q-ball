
\subsection{Apply TV to the SHM coefficients (Ouyang '14)}

\paragraph{Saddle point form}

Again reconstructing from HARDI and regularizing at the same time, Ouyang '14
applies the total variation regularizer to the (vectorial) spherical harmonics
coefficients and combines it with a least-squares fidelity.
Again, we need the spherical harmonics sampling Matrix $\Psi \in \IR^{l,l'}$,
the reconstruction matrix $M \in \IR^{l',l'}$ and we set $f = \log(-\log(E))$,
where $E$ are the HARDI data.
The problem's saddle point form then reads
\begin{align*}
    \min_{u_1,u_2,v} \max_{p} \quad
        & \frac{1}{2} \langle u_2 - f, u_2 - f \rangle_b + \langle Dv, p \rangle \\
    \text{s.t.}\quad 
        & \Psi M v^i = u_2^i, ~\Psi v^i = u_1^i ~\forall i, \\
        & u_1^i \geq 0, ~\langle u_1^i, b \rangle = 1 ~\forall i, \\
        & \|p^{i}\|_{2} \leq \lambda ~\forall i
\end{align*}
or, using more variables and less constraints,
\begin{align*}
    \min_{u_1,u_2,v} \max_{p,q_0,q_1,q_2} \quad
        & \frac{1}{2} \langle u_2 - f, u_2 - f \rangle_b
            + \langle Dv, p \rangle
            + \sum_{i} q_0^i \cdot (b^T u_1^i - 1) \\
        &\quad + \sum_{i} \langle q_1^i, \Psi v^i - u_1^i \rangle
            + \sum_{i} \langle q_2^i, \Psi M v^i - u_2^i \rangle \\
    \text{s.t.}\quad 
        & u_1^i \geq 0, ~\|p^{i}\|_{2} \leq \lambda ~\forall i.
\end{align*}
Here, we denote the Frobenius- or euclidean norm by $\|\cdot\|_{2}$.

\paragraph{Primal and dual objectives}

Accordingly, the primal formulation of the problem is
\begin{align*}
    \min_{u_1,u_2,v} \quad
        & \frac{1}{2} \langle u_2 - f, u_2 - f \rangle_b 
            + \lambda \sum_{i} \| (D v)^{i} \|_{2} \\
    \text{s.t.}\quad 
        & \Psi M v^i = u_2^i, ~\Psi v^i = u_1^i ~\forall i, \\
        & u_1^i \geq 0, ~\langle u_1^i, b \rangle = 1 ~\forall i
\end{align*}
and the dual formulation is
\begin{align*}
    \max_{p,q_0,q_1,q_2} \quad
        & -\sum_i q_0^i+ \sum_{i,k} \frac{b_k}{2} \left [
                \left(f_k^i\right)^2
                - \left(b_k^{-1} q_2^{ik} + f_k^i\right)^2
            \right ] \\
    \text{s.t.}\quad 
        & \|p^{i}\|_{2} \leq \lambda ~\forall i, \\
        & q_0^i b^k - q_1^{ki} \geq 0 ~\forall i, k, \\
        & \Psi^T q_1^i + M \Psi^T q_2^i - (\divergence p)^i = 0 ~\forall i.
\end{align*}

\paragraph{Proximal mappings}

We rewrite the saddle point form as follows:
\begin{align*}
    \min_{u_1,u_2,v} \max_{p,q_0,q_1,q_2} \quad
        & G(u_1,u_2,v) + \langle K(u_1,u_2,v), (p,q_0,q_1,q_2) \rangle - F^*(p,q_0,q_1,q_2),
\end{align*}
where, writing $\beta = \diag(b)$,
\begin{align*}
    G(u_1,u_2,v) &= \frac{1}{2} \langle u_2-f, u_2-f \rangle_b
        + \delta_{\{u_1 \geq 0\}}, \\
    F^*(p,q_0,q_1,q_2) &= \sum_{i} q_0^i 
        + \sum_{i} \delta_{\{\|p^{i}\|_{2} \leq \lambda\}}, \\
    K(u_1,u_2,v) &= (
        Dv,
        b^T u_1,
        \Psi v^i - u_1^i,
        \Psi M v^i - u_2^i
    ), \\
    K^*(p,q_0,q_1,q_2) &= (
        q \otimes b - q_1,
        -q_2,
        \Psi^T q_1^i + M \Psi^T q_2^i - \divergence{p}
    ).
\end{align*}

For a first-order approach we use the proximal mappings
\begin{align*}
    \Prox_{\sigma F*}(\bar{p},\bar{q_0},\bar{q_1},\bar{q_2})
    &= \argmin_{p,q_0,q_1,q_2} \left\{
        \frac{\|
            (p,q_0,q_1,q_2)-(\bar{p},\bar{q_0},\bar{q_1},\bar{q_2})
        \|^2}{2\sigma} + F^*(p,q_0,q_1,q_2)
    \right\} \\
    &= (\proj_{\lambda,2}(\bar{p}),\bar{q_0}-\sigma e,\bar{q_1},\bar{q_2}),
\end{align*}
where $e = (1,\dots,1)$, and
\begin{align*}
    \Prox_{\tau G}(\bar{u_1},\bar{u_2},\bar{v})
    &= \argmin_{u_1,u_2,v,w} \left\{
        \frac{\|
            (u_1,u_2,v)-(\bar{u_1},\bar{u_2},\bar{v})
        \|^2}{2\tau} + G(u_1,u_2,v)
    \right\} \\
    &= \left(
        \max(0,\bar{u_1}),
        (I+\tau \beta)^{-1}(\bar{u_2} + \tau \beta f),
        \bar{v}
    \right).
\end{align*}

\paragraph{The algorithm}

Now we have everything in place to formulate the algorithm:
\begin{align*}
    (p^{k+1},q_0^{k+1},q_1^{k+1},q_2^{k+1}) &= \Prox_{\sigma F^*}(
        (p^{k},q_0^{k},q_1^{k},q_2^{k})
        + \sigma K(\bar{u_1}^{k},\bar{u_2}^{k},\bar{v}^{k})
    ), \\
    (u_1^{k+1},u_2^{k+1},v^{k+1}) &= \Prox_{\tau G}(
        (u_1^{k},u_2^{k},v^{k})
        - \tau K^*(p^{k+1},q_0^{k+1},q_1^{k+1},q_2^{k+1})
    ), \\
    \bar{u_1}^{k+1} &= u_1^{k+1} + \theta(u_1^{k+1}-u_1^{k}), \\
    \bar{u_2}^{k+1} &= u_2^{k+1} + \theta(u_2^{k+1}-u_2^{k}), \\
    \bar{v}^{k+1} &= v^{k+1} + \theta(v^{k+1}-v^{k}).
\end{align*}

