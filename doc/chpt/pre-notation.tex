
\section{Prerequisites}

\subsection{Notation and variables involved}

We assume a $d$-dimensional image domain $\Omega$ that is discretized using
$n$ points $x^1, \dots, x^n \in \Omega$.
Differentiation in $\Omega$ is done on a staggered grid with Neumann boundary
conditions such that the dual operator to the differential operator $D$ is the
negative divergence with vanishing boundary values.
We assume that the image takes values in the space of probability measures on
the two-dimensional sphere $\IS^2$, which is triangulated using $l$ points
$z^1, \dots, z^l \in \IS^2$. 

Integration on $\Omega \times \IS^2$ is discretized as
\begin{equation}\label{eq:discrete-int}
    \langle u, v \rangle_b := \sum_{i,k} b_{k} u_k^i v_k^i
    \vspace{-7.5pt}
\end{equation}
whenever $u,v \in \IR^{n,l}$ are the discretizations of functions on
$\Omega \times \IS^2$, i.\,e. $u_k^i \approx u(x^i,z^k)$.
Equation \eqref{eq:discrete-int} assumes uniform spacing of the $x^i \in \Omega$,
but makes use of a weight vector $b \in \IR^l$ to account for the volume element
at each $z^k \in \IS^2$.

Gradients of functions on $\IS^2$ are defined on a staggered grid of $m$ points
$y^1, \dots, y^m \in \IS^2$ such that each $y^j$ has $3$ neighboring points $
    \nbhd_j \subset \{1, \dots, l\}$, $\#\nbhd_j = 3,
$ among the $z^k$.
The corresponding tangent vectors
\begin{equation}
    v^{j,k} := \exp^{-1}_{y_j}(z^k) \in T_{y_j}\IS^2,
\end{equation}
pointing from $y^j$ in the direction of $z^k$, $k \in \nbhd_j$, are encoded in
the matrix $M^j \in \IR^{3,s}$ such that $
    \langle v^{j,k}, v \rangle = (M^j v)_k
$, whenever $v \in \IR^2 \cong T_{y_j}\IS^2$.

We regard the gradient $g \in \IR^{m,2}$ of a function $p \in \IR^{l}$ on $\IS^2$
as the vector in the tangent space that allows for the best pointwise
approximation of $p$ in an $l^2$-sense:
\begin{equation}\label{eq:approx-grad-on-M}
    g^j = \argmin_{v \in \IR^2} \min_{c \in \IR} \sum_{k \in \nbhd_j} (
        c + \langle v^{j,k}, v \rangle - p^k
    )^2.
\end{equation}
The variable $c$ replaces the value of $p$ at $y^j$ which is unknown since $p$
is discretized on the points $z^k$ only.
The first order optimality conditions for \eqref{eq:approx-grad-on-M} can be
written in the compact form
\begin{equation}
    A^j g^j = B^j P^j p,
\end{equation}
where, for each $j$, the sparse matrix $P^j \in \{0,1\}^{3,l}$ encodes the neighborhood
relations of $y^j$ and $A^j \in \IR^{2,2}$, $B^j \in \IR^{2,3}$ are defined by
\begin{align}
    A^j := B^j M^j, &&
    B^j := {M^j}^T E, &&
    E := (I - r^{-1} ee^T), &&
    e := (1, 1, 1) \in \IR^3.
\end{align}

In the following, the dimensions of the primal and dual variables are 
\begin{align*}
    & u \in \IR^{l,n}, && w \in \IR^{n,m,2,d}, \\
    & p \in \IR^{l,d,n}, && g \in \IR^{n,m,2,d}, && q_0 \in \IR^{n},
        && q_1 \in \IR^{l',n}, && q_2 \in \IR^{l',n}.
\end{align*}
Here, $l'$ denotes the number of labels used in the interim discretization of
the sphere, crucial for the inverse Laplace-Beltrami operator
$\Phi \in \IR^{l',l}$.
The bounds are given by $f_1, f_2 \in \IR^{l',n}$.

\subsection{Inverse Laplace-Beltrami operator}

For the Laplace-Beltrami operator on the sphere $\Delta_b$, there's an inversion
formula based on the fundamental solution,
\[
    U^*(x,x_0) := -\frac{1}{4\pi}\log|1 - x \cdot x_0|
    \quad \forall x, x_0 \in \IS^2,
\]
namely:
Whenever $f \in C^2(\IS^2)$, we have
\[
    \frac{1}{4\pi}\int_{\IS^2} f(x) \,dx
    - \int_{\IS^2} U^*(x,x_0) \Delta_b f(x) \,dx = f(x_0).
\]
In particular, for constants, we have
\begin{align*}
    \int_{\IS^2} U^*(x,x_0) \,dx
    &= \int_{\IS^2} U^*(x,e_3) \,dx \\
    &= -\frac{1}{4\pi} \int_0^{2\pi}\int_0^\pi
            \log|1 - \cos\theta| \sin\theta
        \,d\theta d\phi \\
    &= -\frac{1}{2} \int_0^\pi
            \log|1 - \cos\theta| \sin\theta
        \,d\theta \\
    &= -\frac{1}{2} \int_{-1}^1 \log|1 - y| \,dy \\
    &= -\frac{1}{2} \int_{0}^2 \log|z| \,dz \\
    &= 1 - \log(2).
\end{align*}

\subsection{Inverse problem from Aganj's approach}

Writing $F := \log(-\log(S/S_0))$, the formula due to Aganj reads:
\[
    \ODF = \frac{1}{4\pi} + \Delta_b\left[\FRT(F)\right].
\]
Now, the inversion formula tells us that
\[
    \frac{1}{4\pi}\int_{\IS^2} \FRT(F)(x) \,dx
    - \int_{\IS^2} U^*(x,x_0) (\ODF(x) - \frac{1}{4\pi}) \,dx
    = \FRT(F)(x_0)
\]
or, alternatively,
\[
    \FRT(F)(x_0) - \overline{\FRT(F)}
    = \frac{1 - \log(2)}{4\pi} - U^*[\ODF](x_0).
\]


